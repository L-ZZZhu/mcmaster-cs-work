\documentclass[12pt]{article}
\parindent0pt
\usepackage{ctex}
\usepackage{amsmath}
\usepackage{mathtools,amssymb}
\title{assignemnt}
\author{Longyu Zhu}
\begin{document}

\section*{2.1}
\subsection*{Exercise 12}
(a)\\
$\varnothing  \in \{\varnothing \}$\\
True, for the reason that $\{\varnothing \}$ meaning a empty set set so that $\varnothing$ is a element of empty set.\\

(b)\\
$\varnothing\in \{\varnothing ,\{\varnothing \}\}$\\
True, same with (a) the $\varnothing$ is a element of that set.\\

(c)\\
$\{\varnothing \}\in \{\varnothing \}$\\
False, they are same set that meaning they can not element in that set.\\

(d)\\
$\{\varnothing \}\in \{\{\varnothing \}\}$\\
True, $\{\varnothing\}$ is a element of $\{\{\varnothing\}\}$.\\

(e)\\
$\{\varnothing \}\subset \{\varnothing ,\{\varnothing \}\}$
True, $\{\varnothing\}$ is a element of $\{\varnothing,\{\varnothing\}\}$.\\

(f)\\
$\{\{\varnothing \}\}\subset \{\varnothing ,\{\varnothing \}\}$
True, $\{\{\varnothing\}\}$ can be undersatnd the set $\{\varnothing\}$ so that is subset of the big set.\\

(g)\\
$\{\{\varnothing \}\}\subset \{\{\varnothing \},\{\varnothing \}\}$
False, this euqal can be write like this $\{\{\varnothing \}\}= \{\{\varnothing \},\{\varnothing \}\}$ so that is False.\\

\section*{2.2}
\subsection*{Exercise 14}
$$ 
\left\{
\begin{aligned}
A-B={1,5,7,8)} \\
B-A={2,10}\\
A\cap B={3,6,9}
\end{aligned}
\right.
$$
\\
\begin{equation}\nonumber
    \begin{split}
        A & =(A\cap B)\cup (A-B)\\
           & =\{3,6,9\}\cup \{1,5,7,8\}\\
           & =\{1,3,5,6,7,8,9\}\\
        B & =(A\cap B)\cup (B-A)\\
           & =\{3,6,9\}\cup \{2,10\}\\
           & =\{2,3,6,9,10\}
    \end{split}
\end{equation}
so $A=\{1,3,5,6,7,8,9\}$ and $B=\{2,3,6,9,10\}$\\

\subsection*{Exercise 20}
(a)\\
$(A\cup B)\subseteq (A\cup B\cup C)$\\
\begin{equation}\nonumber
    \begin{split}
    &x\in A \cup B\\
    &x\in A\lor x\in B\\
    &x\in A\lor x\in B\lor x\in C\\
    &x\in A\cup B\cup C\\
    &(A\cup B)\subseteq (A\cup B\cup C)
    \end{split}
\end{equation}

(b)\\
$(A\cap B\cap C)\subseteq (A\cap B)$\\
\begin{equation}\nonumber
    \begin{split}
    &x\in A\cap B\cap C\\
    &x\in A\land x\in B\land x\in C\\
    &x\in A\land x\in B\\
    &x\in A\cap B\\
    &(A\cap B\cap C)\subseteq (A\cap B)
    \end{split}
\end{equation}

(c)\\
$(A-B)-C\subseteq A-C$\\
\begin{equation}\nonumber
    \begin{split}
    &x\in (A-C)\cap (C-B)\\
    &x\in (A-C)\land \lnot (x\in C)\\
    &x\in A\land \lnot (x\in B)\land \lnot (x\in C)\\
    &x\in A\land \lnot (x\in C)\\
    &x\in A-C\\
    &(A-B)-C\subseteq A-C
    \end{split}
\end{equation}

(d)\\
$(A-C)\cap (C-B)=\varnothing $\\
\begin{equation}\nonumber
    \begin{split}
    &x\in (A-C)\cap (C-B)\\
    &x\in (A-C)\land x\in (C-B)\\
    &x\in A\land \lnot (x\in C)\land x\in C\land \lnot (x\in B)\\
    &x\in A\land F\land \lnot (x\in B)\\
    &x\in \varnothing\\
    &\varnothing\subseteq (A-C)\\
    &(A-C)\cap (C-B)=\varnothing
    \end{split}
\end{equation}

(e)\\
$(B-A)\cup (C-A)=(B\cup C)-A$\\
\begin{equation}\nonumber
    \begin{split}
    &x\in(B-A)\cup (C-A)\\
    &x\in(B-A) \lor x\in (C-A)\\
    &(x\in B\land \lnot (x\in A))\lor (x\in C\land \lnot (x\in A))\\
    &(x\in B\lor x\in C)\land \lnot (x\in A)\\
    &(x\in B\lor C)\land \lnot (x\in A)\\
    &x\in(B\cup C)-A\\
    &x\in B\cup C\land \lnot (x\in A)\\
    &(x\in B \lor x\in C)\land \lnot (x\in A)\\
    &(x\in B\land \lnot (x\in A))\lor (x\in C \land \lnot (x\in A))\\
    &x\in B-A\lor x\in C-A\\
    &x\in (B-A)\cup(C-A)\\
    &(B-A)\cup (C-A)=(B\cup C)-A
    \end{split}
\end{equation}

\subsection*{Exercise 48}
\begin{equation}\nonumber
    \begin{split}
    (A\oplus B)\oplus (C\oplus D)=(A\oplus c)\oplus (B\oplus D) 
    &=A\oplus (B\oplus (C\oplus D))\\
    &=A\oplus (B\oplus (D\oplus C))\\
    &=A\oplus ((B\oplus D)\oplus C)\\
    &=A\oplus (C\oplus (B\oplus D))\\
    &=(A\oplus C)\oplus (B\oplus D)
    \end{split}
\end{equation}
\section*{2.3}
\subsection*{Exercise 12}
(a)\\
f(n)=n-1\\
one-to-one, each one number just cna get one result.\\

(b)\\
$f(n)=n^2 +1$\\
not one-to-one, the negitive and positive can be get the same result.like 1 and -1.\\

(c)\\
$f(n)=n^3$\\
one-to-one, each one unumber just can get one result.\\

(d)\\
$f(n)=\lceil n/2\rceil$\\
not one-to-one, when n=0.5 and 1 they can het the same result 1.\\

\subsection*{Exercise 14}
(a)\\
f(m,n)=2m-n\\
Onto, each one can number can get at little one combo of m and n.\\

(b)\\
$f(m,n)=m^2-n^2$\\
Not onto, for example you can not get 2 by this equation.\\

(c)\\
f(m,n)=m+n+1\\
Onto, any number you can get.\\

(d)\\
$f(m,n)=|m|-|n|$\\
Onto, you can get any number even negitive.\\

(e)\\
$f(m,n)=m^2-4$\\
Not onto, you can not get -5 in this equation. or any number less than -5.\\

\subsection*{Exercise 20}
(a)one-to-one but not onto\\
$f(n)=n^2$\\

(b)onto nut not one-to-one\\
$f(n)=\lceil \frac{n}{2}\rceil$\\

(c)both onto and one-to-one(but different from the identity function)\\
$$ 
f(n)=\left \{
\begin{aligned}
n-1  \quad  \text{if n is odd.}\\
n+1   \quad  \text{if n is even.}
\end{aligned}
\right.
$$
\\
(d)neither one to one nor onto\\
f(n)=0\\

\subsection*{Exercise 48}
$[x+\frac{1}{2}]$\\
when x is midway of two integer $[x+\frac{1}{2}]$ will euqal of larger than larger of two integer.\\

\section*{Bonus}
\subsection*{Exercise 74}
if $|A|=|B|$ is one-to-one that meaning $f(a_1)=b_1$\\
if onto B=f(A) B should have all match of A, the function is one-to-one so that will be onto.\\
so f is one-to-one if and only if f is onto when $|A|=|B|$.\\
\end{document}