\documentclass[12pt]{article}
\parindent0pt
\usepackage{ctex}
\usepackage{amsmath}
\usepackage{mathtools,amssymb}
\title{assignemnt 5}
\author{Longyu Zhu}
\begin{document}
\section*{4.1}
\subsection*{Exercise 28}
a)a=-111,m=99\\
a=-111=-197+87=(-2)*99+87=(-2)m+87\\
the quotient m:q=-2\\
the remainder r=87(with $0\leq 87\leq 99$)
a div m=-2\\
a mod m=87\\

b)a=-9999,m=101\\
a=-9999=(-99)*101=(-99)*101+0=(-99)m+0\\
the quotient m:q=-99\\
the remainder r=0 (with $0\leq 0\leq 101$)\\
a div m=-99\\
a mod m=0\\

c)a=10299,m=999\\
a=10299=9990+309=10*999+309=10m+309\\
the quotient m:q=10\\
the remainder r=309 (with $ 0\leq 309\leq 999$)\\
a div m=10\\
a mod m=309\\

d)a=123456,m=1001\\
a=123456=123123+333=123*1001+333=10m+333\\
the quotient m:q=123\\
the remainder r=333(with $0\leq 333\leq 1001$)\\
a div m=123\\
a mod m=333\\

\subsection*{Exercise 30}
a)$a\equiv 43$(mod 23) and $-22\leq a\leq 0$\\
\begin{equation}\nonumber
    \begin{split}
       a&\equiv 43(mod~23)\\
        &\equiv 43-23(mod~23)\\
        &\equiv 20(mod~23)\\
        &\equiv 20-23(mod~23)\\
        &\equiv -3(mod~23)\\
        a=-3
    \end{split}
\end{equation}

b)$a\equiv 17$(mod 29) and $-14\leq a\leq 14$\\
\begin{equation}\nonumber
    \begin{split}
       a&\equiv 17(mod~29)\\
       &\equiv 17-29(mod~29)\\
       &\equiv -12(mod~29)\\
       a=-12
    \end{split}
\end{equation}

c)$a\equiv -11$(mod 21) and $90\leq a\leq 110$\\
\begin{equation}\nonumber
    \begin{split}
       a&\equiv -11(mod~21)\\
       &\equiv -11+21(mod~21)\\
       &\equiv 10(mod~21)\\
       &\equiv 10+21(mod~21)\\
       &\equiv 31(mod~21)\\
       &\equiv 31+21(mod~21)\\
       &\equiv 52(mod~21)\\
       &\equiv 52+21(mod~21)\\
       &\equiv 73(mod~21)\\
       &\equiv 73+21(mod~21)\\
       &\equiv 94(mod~21)\\
       a=94
    \end{split}
\end{equation}

\subsection*{Exercise 36}
a)(177 mod 31 + 270 mod 31)mod 31\\
177 mod 31\\
a=177=155+22=5*31+22=5d+22\\
177 mod 31=22\\
270 mod 31\\
a=270=248+22=8*31+22=8d+22\\
270 mod 31=22\\
\begin{equation}\nonumber
    \begin{split}
        (177~mod~31+270~mod~31)~mod~31&=(22+22)~mod~31\\
        &=44~mod~31\\
        &=44-31~mod~31\\
        &=13~mod~31\\
        &=13
    \end{split}
\end{equation}
\\
b)(177 mod 31 * 270 mod 31)mod 31\\
177 mod 31\\
a=177=155+22=5*31+22=5d+22\\
177 mod 31=22\\
270 mod 31\\
a=270=248+22=8*31+22=8d+22\\
270 mod 31=22
\begin{equation}\nonumber
    \begin{split}
        (177~mod~31*270~mod~31)~mod~31&=(22*22)~mod~31\\
        &=484~mod~31
    \end{split}
\end{equation}
484 mod 31\\
a=484=465+19=15*31+19=15d+19\\
484 mod 31=19\\

\subsection*{Exercise 42}
a,b,c and m are integers $m\geq 2,c>0$ and $a\equiv b$(mod m) then $ac\equiv bc$(mod mc)\\
$a\equiv b$(mod m)\\
m divides a-b\\
m divides c-d\\
a-b=mf\\
c(a-b)=c(mf)\\
ac-bc=(mc)f\\
$ac-bc\equiv(mc)f$\\

\section*{5.1}
\subsection*{Exercise 4}
a) What is the statement P (1)?\\
$P(1):1^3=(\frac{1(1+2)}{2})^2$\\

b) Show that P (1) is true, completing the basis step of the proof of P(n) for all positive integers n.\\
P(1) is true.\\
when$ a=1 1=1^2=1$\\

c)What is the inductive hypothesis of a proof that P(n) is true fpr all positive integers n?\\
$1^3+2^3+...+k^3=(\frac{k(k+1)}{2})^2$\\

d) What do you need to prove in the inductive step of a proof that P(n) is true for all positive integers n?\\
P(k+1)\\

e) Complete the inductive step of a proof that P(n) is true for all positive integers n, identifying where you use the inductive hypothesis.\\
$1^3+2^3+...+k^3+(k+1)^3=(\frac{(k+1)(k+1)+1}{2})^2$\\

f)Explain why these steps show that this formula is true whenever n is a positive integer.\\
Induction\\

\subsection*{Exercise 32}
prove that 3 divides $n^3 +2n$ whenever n is positive integer.\\
n=1\\
$n^3+2n=1^3+2(1)=1+2=3$\\
3 divides $k^3+2k$\\
P(k+1) also true\\
\begin{equation}\nonumber
    \begin{split}
       (k+1)^3+2(k+1)&=k^3+3k^2+3k+1+2k+2\\
       &=k^3+3k^2+5k+3\\
       &=(k^3+2k)+(3k^2+3k+3)\\
       &=(k^3+2k)+3(k^2+k+1)\\
    \end{split}
\end{equation}
$k^3+2k $is divisible by 3 and $3(k^2+k+)$ is divisible by 3, thus $(k+1)^3+2(k+1)$ is then also divisible by 3\\
thus P(k+1) is true\\
3 divides $n^3+2n$ for every positive integer n.\\
\subsection*{Exercise 40}
prove that if $A_1,A_2,...,A_n$ and B are sets, then\\
$(A_1\cap A_2 \cap \cdot \cdot \cdot \cap A_n)\cup B$\\
$\equiv (A_1\cup B)\cap (A_2\cup B)\cap \cdot \cdot \cdot \cap (A_n\cup B)$\\
n=P(k+1)\\
\begin{equation}\nonumber
    \begin{split}
       (A_1\cap  A_2\cap ...\cap A_n)\cup B&=(A_1\cup B)\cap (A_2\cup B)\cap ...\cap (A_n)\cup B)\\
        &=[(A_1\cap A_2\cap ...\cap A_k)\cap A_{k+1}]\cup B\\
        &=[(A_1\cap A_2\cap ...\cap A_k)\cup B]\cap (A_{k+1}\cup B)\\
        &distributive~property~of~union~of~intersection\\
        &=(A_1\cup B)\cap (A_2\cup B)\cap ...\cap (A_k\cup B)\cap (A_{k+1}\cup B)\\
        &since~ the~ result~ is~ true~ for ~k\\
        P(k+1):(A_1\cap A_2\cap ...\cap A_{k+1})\cup B&=(A_1\cup B)\cap (A_2\cup B)\cap ...\cap (A_{k+1}\cup B)\\
        (A_1\cap  A_2\cap ...\cap A_n)\cup B&=(A_1\cup B)\cap (A_2\cup B)\cap ...\cap (A_n)\cup B)\\
    \end{split}
\end{equation}
\end{document}