\documentclass[12pt]{article}
\parindent0pt
\usepackage{ctex}
\usepackage{amsmath}
\usepackage{mathtools,amssymb}
\title{assignemnt 5}
\author{Longyu Zhu}
\begin{document}
\section*{6.1}
\subsection*{Exercise 4}
A particular brand of shirt comes in 12 colors,
has a male version and a female version, 
    and comes in three sizes for each sex.
    How many different types of this shirt are made?\\
12*2*3=72\\
so totally 72 different type of shirts\\
\\
\subsection*{Exercise 16}
How many bit strings of length n, where n is a positive integer, start and end with 1s?\\
for the strings only have two choice 0 and 1\\
so we can get $1*2*2*...*2*1=2^{n-2}$ when $n\geq 2$\\
so we also can get if n=1 just have one chioce can have.\\
\\
\subsection*{Exercise 60}
The International Telecommunications Union (ITU) specifies that a telephone number must consist 
    of a country code with between 1 and 3 digits, except that the code 0 is not available for use as a country code, 
    followed by a number with at most 15 digits. How many available possible telephone numbers are there that satisfy these restrictions?\\
if have three at the begin we will have 9*10*10, if two 9*10, if one just 9\\
so we have 9*10*10+9*10+9=990 kind of choice at the begin\\
for the followed number we have $10^{15}~10^{14}~10^{13}~10^{12}~10^{11}~10^{10}~10^{9}~10^8~10^7~10^6~10^5~10^4~10^3~10^2~10^1$\\
we need plus then $\sum_{n = 1}^{15}10^n=1111111111111110$ \\
at the end we need 990*1111111111111110=1109999999999998890\\
\\
\section*{6.2}
\subsection*{Exercise 6}
There are six professors teaching the introductory discrete mathematics class at a university. 
   The same final exam is given by all six professors. If the lowest possible score on the final is 0 and the highest possible score is 100,
   how many students must there be to guarantee that there are two students with the same professor who earned the same final examination score?\\
at first we have 100+1 kind score can have\\
if we want to get the same score we should have one more that meaning we need at least 102 for same professor.\\
but we have 6 professors here so we need 101*6 to make make sure all professor have 0 to 100 score\\
we just need add one more thus the answers will get that is (6*101)+1=607\\
\\
\subsection*{Exercise 18}
a) Show that if seven integers are selected from the first 10 positive integers, there must be at least two pairs of these integers with the sum 11.\\
we can try for 1,10   2,9   3,8   4,7    5,6\\
thus we can just we at least have two pair available\\
\\
b) Is the conclusion in part (a) true if six integers are selected rather than seven?\\
no, for the reason that the if we get 4 number we just can get at least 1 pair more than 11.\\
\\
\subsection*{Exercise 34}
Show that if there are 100,000,000 wage earners in the United States who earn less than 1,000,000 dollars (but at least a penny), 
then there are two who earned exactly the same amount of money, to the penny, last year.\\
1000000 dollars = 100000000 penny\\
that meaning have 99999999 kind of wage\\
so must have too people get same wage\\
\\
\section*{6.3}
\subsection*{Exercise 6}
a) C(5,1)\\
$\frac{5!}{4!}=\frac{5*4!}{4!}=5$\\
b) C(5,3)\\
$=\frac{5!}{3!(5-3)!}=\frac{5!}{3!2!}=10$\\
c) C(8,4)\\
$=\frac{8!}{4!(8-4)!}=\frac{8!}{4!4!}=70$\\
d) C(8,8)\\
$=\frac{8!}{8!(8-8)!}=\frac{8!}{8!0!}=1$\\
e) C(8,0)\\
$=\frac{8!}{0!(8-0)!}=\frac{8!}{0!8!}=1$\\
f) C(12,6)\\
$=\frac{12!}{6!(12-6)!}=\frac{12!}{6!6!}=924$\\
\subsection*{Exercise 18}
a) are there in total?\\
$2^8=256$\\
\\
b) contain exactly there heads?\\
$C(8,3)=\frac{8!}{3!(8-3)!}=\frac{8!}{3!5!}=56$\\
\\
c) contain at least three heads?\\
C(8,3)=56 C(8,4)=70 C(8,5)=56 C(8,6)=28 C(8,7)=8 C(8,8)=1\\
56+70+56+28+8+1=219\\
\\
d)contain the same number of heads and tails?\\
C(8,4)=70\\
\\
\subsection*{Exercise 32}
a) How many ways are there to select a committee of five members of the department if at least one woman must be on the committee?\\
C(16,5)=$\frac{16!}{5!(16-5)!}=\frac{16!}{5!11!}=4368$(from all people get 5)\\
C(9,5)=$\frac{9!}{5!(9-5)!}=\frac{9!}{5!4!}=126$(how many kind of all man)\\
4368-126=4242\\
\\
b) How many ways are there to select a committee of five members of the department if at least one woman and at least one man must be on the committee?\\
C(7,5)=$\frac{7!}{5!(7-5)!}=\frac{7!}{5!2!}=21$(all for women)\\
4368-126-21=4221\\
\\
\section*{6.4}
\subsection*{Exercise 2}
$(x+y)^5$\\
a) using combinatorial reasoning\\
\begin{equation}\nonumber
    \begin{split}
        (x+u)^5&=(x+y)(x+y)(x+y)(x+y)(x+y)\\
        &=(x^2+xy+xy+y^2)(x^2+xy+xy+y^2)(x+y)\\
        &=(x^2+2xy+y^2)(x^2+2xy+y^2)(x+y)\\
        &=(x^4+2x^3y+x^2y^2+2x^3y+4x^2y^2+2xy^3+x^2y^2+2xy^3+x^2y^2+2xy^3+y^4)(x+y)\\
        &=(x^4+4x^3y+6x^2y^2+4xy^3+y^4)(x+y)\\
        &=x^5+4x^4y+6x^3y^2+4x^2y^3+xy^4+x^4y+4x^3y^2+6x^2y^3+4xy^4+y^5\\
        &=x^5+5x^4y+10x^3y^2+10x^2y^3+5xy^4+y^5
    \end{split}
\end{equation}
\\
b) using the binomial theorem.\\
\begin{equation}\nonumber
    \begin{split}
        (x+y)^5&=\left(\substack{5\\0}\right)x^5y^0+\left(\substack{5\\1}\right)x^4y^1+\left(\substack{5\\2}\right)x^3y^2+\left(\substack{5\\3}\right)x^2y^3+\left(\substack{5\\4}\right)x^1y^4+\left(\substack{5\\5}\right)x^0y^5\\
        &=\frac{5!}{0!5!}x^5y^0+\frac{5!}{1!4!}x^4y^1+\frac{5!}{2!3!}x^3y^2+\frac{5!}{3!2!}x^2y^3+\frac{5!}{4!1!}x^1y^4+\frac{5!}{5!0!}x^0y^5\\
        &=1x^5+5x^4y+10x^3y^2+10x^2y^3+5xy^4+y^5\\
        &=x^5+5x^4y+10x^3y^2+10x^2y^3+5xy^4+y^5
    \end{split}
\end{equation}
\\
\subsection*{Exercise 12}
$(5x^2+2y^3)^6$  $x^ay^b$\\\\
a)a=6,b=9\\
$(\substack{6\\3})5x^{2^2}2y^{3^3}=\frac{6!}{3!3!}5x^{2^2}2y^{3^3}=120*5^3*2^3x^6y^9=20000x^6y^9$\\
\\
b)a=2,b=15\\
$(\substack{6\\1})5x^{2^1}2y^{3^5}=\frac{6!}{1!5!}5x^{2^1}2y^{3^5}=6*5*2^5x^2y^{15}=960x^2y^{15}$\\
\\
c)a=3,b=12\\
not available for this equation.\\
\\
d)a=12,b=0\\
$(\substack{6\\0})5x^{2^6}=\frac{6!}{0!6!}5x^{2^6}=5^6x^{12}=15625x^{12}$\\
\\
e)a=8,b=9\\
not available for this equation.\\
\end{document}